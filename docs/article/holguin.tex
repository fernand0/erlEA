\documentclass{article}
\usepackage[ansinew]{inputenc}
\usepackage{graphicx}
\usepackage{color}

\title{dAE2rl: experimentaci�n de algoritmos evolutivos distribuidos con Erlang}

\author{MSc. Jos� Albert Cruz Almaguer \\
Dr. Juan Juli�n Merelo Guerv�s\\
Dr, Rodolfo Garc�a Berm�dez}

\begin{document}

\maketitle

\begin{abstract}
Los algoritmos gen�ticos (AGs) constituyen una de las metaheur�sticas
m�s utilizadas hoy d�a para obtener soluciones en tiempos razonables a
los problemas m�s complejos que enfrenta la Ciencia de la
Computaci�n. Para problemas especialmente complejos o simplemente para
aprovechar la capacidad computacional de instalaciones permanentes
(como aulas u ordenadores instalados en oficinas) los algoritmos
gen�ticos paralelos son un tipo de AG en los que las soluciones
tentativas son evaluadas y evolucionan en paralelo, lo que permite una
mayor capacidad de evaluaci�n simult�nea pero, a la vez, tiene
ventajas desde el punto de vista del AG.
Aunque existen herramientas
para el trabajo con este tipo de algoritmos no es com�n encontrar
alguna que explote en alto grado alg�n paradigma de programaci�n
distribuido. Erlang es un lenguaje de programaci�n con soporte
integrado para la concurrencia y la distribuci�n, ha sido en m�s de
una ocasi�n escogido por encima de C/C++ para el desarrollo de
sistemas por su facilidad para expresar procesos concurrentes. En el
presente trabajo presentamos dAE2rl, una herramienta
desarrollada fundamentalmente en Erlang para la experimentaci�n con AG
distribuidos.  Haremos medidas sobre el mismo en diferentes tipos de
sistemas y mostraremos que los resultados obtenidos permiten
aprovechar mucho mejor la capacidad computacional de instalaciones
existentes.
\end{abstract}


\nocite{*}
\bibliographystyle{plain}
\bibliography{referencias}

\end{document}
