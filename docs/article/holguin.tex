\documentclass{article}
\usepackage[ansinew]{inputenc}
\usepackage{graphicx}
\usepackage{color}

\title{dAE2rl: experimentaci�n de algoritmos evolutivos distribuidos con Erlang}

\author{MSc. Jos� Albert Cruz Almaguer \\
Dr. Juan Juli�n Merelo Guerv�s}

\begin{document}

\maketitle

\begin{abstract}
Los algoritmos gen�ticos (GA) constituyen una de las metaheur�sticas m�s utilizas hoy d�a para obtener soluciones en tiempos razonables a los problemas m�s complejos que enfrenta la Ciencia de la Computaci�n. Los algoritmos gen�ticos paralelos son un tipo de GA en los que las soluciones tentativas son evaluadas y evolucionan en paralelo. Aunque existen herramientas para el trabajo con este tipo de algoritmos no es com�n encontrar alguna que explote en alto grado alg�n paradigma de programaci�n distribuido. Erlang es un lenguaje de programaci�n con soporte integrado para la concurrencia y la distribuci�n, ha sido en m�s de una ocasi�n escogido por encima de C/C++ para el desarrollo de sistemas y en el presente trabajo nos disponmos a demostrar su idoneidad.
\end{abstract}


\nocite{*}
\bibliographystyle{plain}
\bibliography{referencias}

\end{document}
